\documentclass[journal]{IEEEtran}

\usepackage{hyperref}

% cspell: disable
% Using Grammarly instead

\immediate\write18{texcount -inc -sum -1 assignment.tex > wordcount.tmp}
\newcommand\wordcount{
    \input{wordcount.tmp}
}

\begin{document}

% Cover sheet
%TC: ignore
{\Large \textbf{Cover Sheet}}

Kieran Knowles

w20013000

Computer Science

Intelligent Systems KF5042

Submission date: \today

Word count: \wordcount

%TC: endignore

\title{How the emphasis of sentiments in training data can impact the performance of a sentiment analysis model}
\author{Kieran Knowles}
\maketitle

\begin{abstract}
    //TODO: This

\end{abstract}

\section{Introduction}
//TODO: This

\section{Literature Review}
//TODO: This

\section{Methods}
\subsection{Training Data}
Dialogue from the game The Elder Scrolls IV: Oblivion was used as training data for the model. //TODO: Why?
This game is known for its dialogue often being overacted, in part due to its actors being given
their lines in alphabetical order. //TODO: Source

The idea for using a video game as a corpus came from Hämäläinen et al., \cite{hamalainen_video_2022}
In which the game Fallout New Vegas was used as training data with accuracies that
"... are rather low, but they are
in line with the usual accuracies obtained in similar multilabel NLP
classification tasks" \cite[sec. 6]{hamalainen_video_2022}

Oblivion uses the same system in which every line of dialog is annotated with a sentiment
and the game is voiced in the same five languages.
//TODO: This
\subsection{Considerations}
//TODO: This. Lack of actor variety

\subsection{Data Extraction}

\section{Results and Discussion}
//TODO: This

\section{Conclusion and future work}
//TODO: This

%TC: ignore

\bibliographystyle{IEEEtran}
\bibliography{assignment}

%TC: endignore

\end{document}
