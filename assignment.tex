\documentclass[journal]{IEEEtran}

\usepackage{csvsimple}
\usepackage{hyperref}

% cspell: disable
% Using Grammarly instead

% https://tex.stackexchange.com/questions/534/is-there-any-way-to-do-a-correct-word-count-of-a-latex-document
\immediate\write18{texcount -inc -sum -1 assignment.tex > wordcount.tmp}
\newcommand\wordcount{
    \input{wordcount.tmp}
}

\begin{document}

% Cover sheet
%TC: ignore
{\Large \textbf{Cover Sheet}}

Kieran Knowles

w20013000

Computer Science

Intelligent Systems KF5042

Submission date: \today

Word count: \wordcount

\title{A Comparative Study of how the Emphasis of Sentiments in Training Data can Impact the Performance of a Sentiment Analysis Model}
\author{Kieran Knowles}
\maketitle

%TC: endignore

\begin{abstract}
    //TODO: This

\end{abstract}

\section{Introduction}
This paper aims to study how the emphasis of sentiments in training data can impact the performance of a sentiment analysis
model. The hypothesis is that having more emphasised emotions in training data can improve the performance of the resulting
model, even when running on less emphasised speech.

While computer sentiment analysis has existed for decades \cite{stone_computer_1963}, existing research has primarily focused on
its use on text rather than speech. //TODO: Source

This review is incomplete however as it doesn't cover the impact of emphasis of emotions in training data.
Whilst researching this subject, no papers on the subject were found.
This could be due to the fact that "audio sentiment analysis is still in a nascent stage
in the research community". \cite{maghilnan_sentiment_2017}

//TODO: This

\section{Literature Review}

//TODO: This

As the purpose of this study was to determine how the emphasis of speech affects
the performance of a model, only the audio from the training data was used as input,
for the model despite hybrid models being more accurate. \cite{bhaskar_hybrid_2015}

\subsection{Algorithm Selection}
//TODO: This

\subsection{Training Data}
Video games have previously been used as training data for sentiment analysis models, such as in
Hämäläinen et al.'s paper, \cite{hamalainen_video_2022} which used Fallout New Vegas as training data.

This study uses the games Fallout New Vegas and The Elder Scrolls IV: Oblivion as training data as
Oblivion uses the same system in which every line of dialogue is annotated with one of seven sentiments
(anger, disgust, fear, happy, neutral, sad, and surprised) and the game is voiced in the same five languages
(English, French, Italian, German, and Spanish).

Oblivion is known for sometimes having over-emphasised emotions in its dialogue.
This over-emphasis may be due to the fact that the game's voice actors were given their lines
in alphabetical order. \cite[1:00]{noclip_-_video_game_documentaries_music_2018}

Fallout New Vegas was chosen as a second dataset as it uses a similar system, which
allows for the same code to be used to extract the data from both games.

\section{Methods}
\subsection{Considerations}
//TODO: This. Lack of actor variety, copyright (maybe in results and discussion)

Some lines of dialogue have actor notes which provided additional context to
the voice actors. These notes may lead actors to show a sentiment that does not
match the emotion annotated in the dialogue itself.

Additionally, the neutral emotion is the default emotion for any new line in the editor.
Because of this, some lines were never annotated with a sentiment despite the recorded
dialogue showing a specific emotion. See table \ref{table:bad_annotation}

\begin{table}[h]
    \begin{tabular}{| c | l |}
        \hline
        Text & Don't try to manipulate me. \\ \hline
        Annotation & Neutral \\ \hline
        File & oblivion.esm\textbackslash nord\textbackslash f\textbackslash generic\textunderscore admirehate\textunderscore 00062311\textunderscore 1.mp3 \\ \hline
        Actual emotion & Anger \\ \hline
    \end{tabular}
    \caption{An example of a lines annotation not matching the recording}
    \label{table:bad_annotation}
\end{table}

\subsection{Data Extraction}
Due to the instability of Fallout New Vegas' official tools as reported by Hämäläinen et al., \cite{hamalainen_video_2022},
which was assumed to be the same for Oblivion, the data was instead extracted using a custom script
written using the Mutagen library. \cite{noauthor_mutagen_2023}
This script outputs a CSV file containing the dialogue, the sentiment, and its extracted file path.

See tables \ref{table:category_counts_oblivion} and \ref{table:category_counts_new_vegas} for a breakdown of the number of samples in each category.

The data extraction script outputs a CSV file that can then be imported into MATLAB.

//TODO: This

\begin{table}[h]
    \begin{tabular}{| c | c |}
        \hline
        Category & Sample Count
        \csvreader[head to column names]{src/out/category_counts_oblivion.csv}{}%
            {\\ \hline \Name & \Count}%
        \\ \hline
    \end{tabular}
    \caption{The number of samples in each category for Oblivion}
    \label{table:category_counts_oblivion}
\end{table}

% \begin{table}[h]
%     \begin{tabular}{| c | c |}
%         \hline
%         Category & Sample Count
%         \csvreader[head to column names]{src/out/category_counts_new_vegas.csv}{}%
%             {\\ \hline \Name & \Count}%
%         \\ \hline
%     \end{tabular}
%     \caption{The number of samples in each category for New Vegas}
%     \label{table:category_counts_new_vegas}
% \end{table}

\subsection{Source Code}
//TODO: How to include this?

\section{Results and Discussion}
//TODO: This

\section{Conclusion and future work}
//TODO: This

%TC: ignore

\bibliographystyle{IEEEtran}
\bibliography{assignment}

%TC: endignore

\end{document}
